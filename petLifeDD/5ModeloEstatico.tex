% !TeX root = proyecto.tex

%=========================================================
\chapter{Modelo Estático}
\label{cap:modelo_estatico}

En este capítulo se describe la estructura interna del sistema **PetLife**. A diferencia del modelo arquitectónico, que se enfoca en nodos y servidores, el modelo estático detalla cómo está organizado el código fuente, la agrupación de archivos en paquetes lógicos y las relaciones entre las clases y entidades que conforman la solución.

%---------------------------------------------------------
\section{Descripción general del modelo estático}

El diseño estático de PetLife refleja la naturaleza híbrida del proyecto. El código se encuentra particionado en dos grandes subsistemas lógicos que coinciden con la estructura de directorios del proyecto:

\begin{enumerate}
    \item \textbf{Subsistema Web (PHP):} Encargado de la gestión de usuarios, mascotas y la interfaz gráfica. Sigue una estructura modular donde cada script o archivo cumple una función específica (Vista o Controlador).
    \item \textbf{Subsistema de IA (Python):} Organizado bajo patrones de diseño de microservicios, utilizando routers para los puntos de entrada y clases para la lógica de predicción.
\end{enumerate}

Esta separación permite que el mantenimiento del código sea ordenado, evitando la mezcla de lenguajes y responsabilidades en un mismo directorio.

%---------------------------------------------------------
\section{Diagrama de paquetes}

El diagrama de paquetes muestra la organización de alto nivel del código fuente, agrupando los elementos en subsistemas lógicos según su funcionalidad.

\begin{figure}[h!]
    \centering
    %-------------------------------------------------------
    % INSTRUCCIONES PARA TU DIBUJO (Diagrama de Paquetes):
    % Debes dibujar "Carpetas" (símbolo de paquete UML).
    % Dibuja un Paquete Grande llamado "PetLife System" y adentro pon estos paquetes hijos:
    % 1. Paquete "Vista/Interfaz" (Representa tu carpeta 'estilo' y 'recursosVisuales').
    % 2. Paquete "Lógica Web" (Representa tu carpeta 'recursosWeb' y archivos .php raíz).
    % 3. Paquete "API Service" (Representa tu carpeta 'petlife_api').
    % 4. Paquete "Persistencia" (Representa tu BD y archivos de conexión).
    %
    % Dibuja flechas punteadas ("import" o "use") entre ellos:
    % - Vista --> usa --> Lógica Web
    % - Lógica Web --> usa --> API Service
    % - Lógica Web --> usa --> Persistencia
    %-------------------------------------------------------
    
    \framebox[0.9\textwidth]{\rule{0pt}{8cm} [Insertar aquí el Diagrama de Paquetes UML]}
    
    \caption{Diagrama de paquetes del sistema, reflejando la estructura de directorios y dependencias.}
    \label{fig:diagrama_paquetes}
\end{figure}

%---------------------------------------------------------
\section{Explicación del diagrama de paquetes}

La organización mostrada en la Figura \ref{fig:diagrama_paquetes} corresponde directamente con la estructura de directorios del proyecto, facilitando la navegación para los desarrolladores:

\subsection{Paquete de Interfaz (Frontend)}
Agrupa todos los recursos estáticos encargados de la presentación visual. Corresponde a los directorios \texttt{estilo/} (CSS) y \texttt{js/} (jQuery). Este paquete no contiene lógica de negocio, solo reglas de visualización y comportamiento del lado del cliente.

\subsection{Paquete de Lógica Web (Backend PHP)}
Contiene los scripts principales del sistema alojados en \texttt{recursosWeb/}. Este paquete actúa como el controlador del sistema, recibiendo las peticiones del usuario, validando los datos y coordinando las respuestas. Incluye archivos clave como \texttt{user.php} y \texttt{mis\_mascotas.php}.

\subsection{Paquete de Servicios de IA (API Python)}
Encapsula toda la lógica de Inteligencia Artificial dentro del directorio \texttt{petlife\_api/}. Este paquete es autónomo y contiene sus propios sub-paquetes:
\begin{itemize}
    \item \textbf{Routers:} Define los puntos de entrada (endpoints) de la API.
    \item \textbf{Models:} Contiene los archivos binarios del modelo predictivo (\texttt{.joblib}).
\end{itemize}

\subsection{Paquete de Datos (Persistencia)}
Representa la capa de almacenamiento. Incluye los scripts de configuración de base de datos (\texttt{config.php}, \texttt{database.py}) y la estructura de directorios \texttt{uploads/} donde se almacenan las imágenes de las mascotas.

%---------------------------------------------------------
\section{Diagramas de clases}

Los diagramas de clases detallan la estructura de las entidades del sistema, sus atributos y métodos, así como las relaciones entre ellas. Dado que el sistema utiliza una base de datos relacional y un servicio de objetos en Python, se representan las clases principales de ambos contextos.

\begin{figure}[h!]
    \centering
    %-------------------------------------------------------
    % INSTRUCCIONES PARA TU DIBUJO (Diagrama de Clases):
    % Aquí dibuja las "Clases" basadas en tu SQL y tu lógica.
    %
    % Clase 1: Usuario
    % - Atributos: id, nombre, email, password_hash
    % - Métodos: registrar(), iniciarSesion(), cerrarSesion()
    %
    % Clase 2: Mascota
    % - Atributos: id, nombre, raza, edad, foto_path, id_usuario
    % - Métodos: crear(), obtenerHistorial(), predecirVida()
    %
    % Clase 3: PrediccionIA (Representa tu lógica Python)
    % - Atributos: modelo_cargado
    % - Métodos: cargarModelo(), realizarInferencia(datos)
    %
    % Relaciones:
    % - Usuario "1" --- tien "0..*" ---> Mascota (Agregación o Asociación)
    % - Mascota "1" --- usa ---> PrediccionIA (Dependencia)
    %-------------------------------------------------------
    
    \framebox[0.9\textwidth]{\rule{0pt}{10cm} [Insertar aquí el Diagrama de Clases UML]}
    
    \caption{Diagrama de clases principales del dominio de PetLife.}
    \label{fig:diagrama_clases}
\end{figure}

%---------------------------------------------------------
\section{Explicación de los diagramas de clases}

El modelo de clases representa las entidades fundamentales del negocio y cómo interactúan para cumplir los requisitos funcionales. A continuación se describen las clases más relevantes ilustradas en la Figura \ref{fig:diagrama_clases}:

\subsection{Clase Usuario}
Representa a los actores que interactúan con el sistema (Dueños o Veterinarios).
\begin{itemize}
    \item \textbf{Responsabilidad:} Gestionar la autenticación y los datos personales.
    \item \textbf{Atributos clave:} Identificador único, credenciales cifradas y datos de contacto.
    \item \textbf{Relaciones:} Tiene una relación de uno a muchos ($1..*$) con la clase \textbf{Mascota}, indicando que un usuario puede ser propietario de múltiples animales.
\end{itemize}

\subsection{Clase Mascota}
Es la entidad central del sistema, sobre la cual giran la mayoría de las operaciones clínicas.
\begin{itemize}
    \item \textbf{Responsabilidad:} Almacenar la información biológica del animal (raza, edad, peso) y referencias a sus archivos multimedia (fotos).
    \item \textbf{Métodos principales:} Incluye lógica para registrar nuevos ingresos y solicitar análisis al módulo de IA.
\end{itemize}

\subsection{Clase ServicioPredictivo (Módulo IA)}
Esta clase es una abstracción del componente de Python encargado de la Inteligencia Artificial.
\begin{itemize}
    \item \textbf{Responsabilidad:} Cargar el modelo \texttt{petlife\_model.joblib} en memoria y exponer el método de inferencia.
    \item \textbf{Interacción:} No persiste datos en la base de datos directamente; su función es procesar los atributos de la clase \textbf{Mascota} y retornar un valor calculado (ej. esperanza de vida estimada), que posteriormente es mostrado en la interfaz.
\end{itemize}