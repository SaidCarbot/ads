% !TeX root = proyecto.tex

%=========================================================
\chapter{Propiedades de Software}
\label{cap:propiedades}

En el presente capítulo se establecen los Requerimientos No Funcionales (RNF) que rigen el comportamiento y la calidad del sistema \textbf{PetLife}. A partir de la investigación realizada y basándose en el modelo de calidad de la norma \textbf{ISO/IEC 25010} y la metodología \textit{Architecture-Centric Design Method} (ACDM), se han definido las tácticas arquitectónicas necesarias. Dado que el proyecto implementa una arquitectura híbrida (interfaz web en PHP y módulo de Inteligencia Artificial en Python), el diseño contempla estrategias específicas para la integración y estabilidad de ambos entornos.

%---------------------------------------------------------
\section{Descripción de los requerimientos no funcionales}

A continuación, se detallan las propiedades de software identificadas como críticas para el proyecto, organizadas por atributo de calidad.

%---------------------------------------------------------
\subsection{Seguridad}

La seguridad del sistema se orienta a la protección de la integridad de los datos clínicos y la privacidad de la información de los usuarios. El diseño incorpora mecanismos de defensa en profundidad tanto en la capa de aplicación web como en la interfaz de programación (API) del modelo predictivo.

\begin{table}[h!]
\centering
\small
\renewcommand{\arraystretch}{1.5}
\begin{tabular}{|p{0.45\textwidth}|p{0.45\textwidth}|}
\hline
\textbf{Requerimiento (RNF)} & \textbf{Solución en el Diseño (Táctica)} \\
\hline
RNF-SEG-01 & Uso de sentencias preparadas en PHP para prevenir inyecciones SQL. \\
\hline
RNF-SEG-02 & Hashing seguro de contraseñas mediante \texttt{password\_hash()}. \\
\hline
RNF-SEG-03 & Validación de sesiones activas antes de permitir acceso a módulos protegidos. \\
\hline
RNF-SEG-04 & Validación de tipo MIME y renombramiento seguro de archivos de imagen. \\
\hline
RNF-SEG-05 & Escapado de caracteres HTML para prevenir ataques XSS. \\
\hline
RNF-SEG-06 & Validación de datos de entrada en la API mediante esquemas Pydantic. \\
\hline
\end{tabular}
\caption{Propiedades de Seguridad y tácticas aplicadas}
\label{tab:rnf_seguridad}
\end{table}

%---------------------------------------------------------
\subsection{Rendimiento}

El rendimiento constituye un factor crítico para la experiencia del usuario, especialmente en el procesamiento de imágenes y la ejecución del modelo de Inteligencia Artificial.

\begin{table}[h!]
\centering
\small
\renewcommand{\arraystretch}{1.5}
\begin{tabular}{|p{0.45\textwidth}|p{0.45\textwidth}|}
\hline
\textbf{Requerimiento (RNF)} & \textbf{Solución en el Diseño (Táctica)} \\
\hline
RNF-PERF-01 & Carga única del modelo de IA en memoria al iniciar el servidor. \\
\hline
RNF-PERF-02 & Compresión automática de imágenes a formato WebP. \\
\hline
RNF-PERF-03 & Comunicación asíncrona mediante AJAX sin recarga de página. \\
\hline
RNF-PERF-04 & Indexación de campos críticos en la base de datos MySQL. \\
\hline
RNF-PERF-05 & Minificación y caché de recursos estáticos CSS y JavaScript. \\
\hline
\end{tabular}
\caption{Estrategias de Rendimiento}
\label{tab:rnf_rendimiento}
\end{table}

%---------------------------------------------------------
\subsection{Disponibilidad}

La disponibilidad garantiza la continuidad operativa del sistema ante fallos parciales de servicios o infraestructura.

\begin{table}[h!]
\centering
\small
\renewcommand{\arraystretch}{1.5}
\begin{tabular}{|p{0.45\textwidth}|p{0.45\textwidth}|}
\hline
\textbf{Requerimiento (RNF)} & \textbf{Solución en el Diseño (Táctica)} \\
\hline
RNF-DISP-01 & Manejo de excepciones ante fallos del servicio de IA. \\
\hline
RNF-DISP-02 & Uso de transacciones SQL para preservar integridad de datos. \\
\hline
RNF-DISP-03 & Endpoint de verificación de estado del servicio de IA. \\
\hline
RNF-DISP-04 & Copias de seguridad periódicas de archivos multimedia. \\
\hline
RNF-DISP-05 & Centralización de configuraciones y credenciales. \\
\hline
\end{tabular}
\caption{Estrategias de Disponibilidad}
\label{tab:rnf_disponibilidad}
\end{table}

%---------------------------------------------------------
\subsection{Escalabilidad}

La escalabilidad permite al sistema crecer en funcionalidades y volumen de datos sin afectar su estabilidad.

\begin{table}[h!]
\centering
\small
\renewcommand{\arraystretch}{1.5}
\begin{tabular}{|p{0.45\textwidth}|p{0.45\textwidth}|}
\hline
\textbf{Requerimiento (RNF)} & \textbf{Solución en el Diseño (Táctica)} \\
\hline
RNF-ESC-01 & Desacoplamiento entre frontend PHP y API de IA en Python. \\
\hline
RNF-ESC-02 & Arquitectura modular para nuevos tipos de predicción. \\
\hline
RNF-ESC-03 & Organización jerárquica del almacenamiento de imágenes. \\
\hline
RNF-ESC-04 & Base de datos relacional normalizada para múltiples usuarios. \\
\hline
RNF-ESC-05 & Separación entre lógica de negocio y presentación. \\
\hline
\end{tabular}
\caption{Estrategias de Escalabilidad}
\label{tab:rnf_escalabilidad}
\end{table}

%---------------------------------------------------------
\subsection{Usabilidad}

La usabilidad busca una interacción clara e intuitiva para veterinarios y propietarios de mascotas.

\begin{table}[h!]
\centering
\small
\renewcommand{\arraystretch}{1.5}
\begin{tabular}{|p{0.45\textwidth}|p{0.45\textwidth}|}
\hline
\textbf{Requerimiento (RNF)} & \textbf{Solución en el Diseño (Táctica)} \\
\hline
RNF-USA-01 & Diseño visual consistente mediante hojas de estilo comunes. \\
\hline
RNF-USA-02 & Retroalimentación inmediata mediante validaciones asíncronas. \\
\hline
RNF-USA-03 & Diseño responsivo con media queries CSS. \\
\hline
RNF-USA-04 & Representación gráfica clara de resultados del modelo de IA. \\
\hline
RNF-USA-05 & Navegación persistente entre módulos principales. \\
\hline
\end{tabular}
\caption{Estrategias de Usabilidad}
\label{tab:rnf_usabilidad}
\end{table}

%---------------------------------------------------------
\subsection{Mantenibilidad}

La mantenibilidad permite corregir, adaptar y evolucionar el sistema con bajo costo técnico.

\begin{table}[h!]
\centering
\small
\renewcommand{\arraystretch}{1.5}
\begin{tabular}{|p{0.45\textwidth}|p{0.45\textwidth}|}
\hline
\textbf{Requerimiento (RNF)} & \textbf{Solución en el Diseño (Táctica)} \\
\hline
RNF-MANT-01 & Uso de tipado y documentación automática en la API. \\
\hline
RNF-MANT-02 & Gestión explícita de dependencias en archivos de configuración. \\
\hline
RNF-MANT-03 & Centralización de la lógica de acceso a datos. \\
\hline
RNF-MANT-04 & Organización semántica de la estructura de directorios. \\
\hline
RNF-MANT-05 & Implementación de mecanismos de logging detallado. \\
\hline
\end{tabular}
\caption{Estrategias de Mantenibilidad}
\label{tab:rnf_mantenibilidad}
\end{table}

%=========================================================
% BIBLIOGRAFÍA
%=========================================================
\begin{thebibliography}{9}

\bibitem{iso25010}
International Organization for Standardization. (2011).
\textit{ISO/IEC 25010: Systems and software quality models}.

\bibitem{acdm}
Lattanze, A. J. (2009).
\textit{Architecting Software Intensive Systems}.
Auerbach Publications.

\bibitem{cleanarch}
Martin, R. C. (2017).
\textit{Clean Architecture}.
Prentice Hall.

\bibitem{fastapi}
Ramírez, S. (2024).
\textit{FastAPI Documentation}.

\bibitem{php}
The PHP Group. (2024).
\textit{PHP Manual: Security and Database Access}.

\end{thebibliography}
