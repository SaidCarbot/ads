% !TeX root = proyecto.tex

%=========================================================
%\chapter{Project Charter}

\newcommand{\ESCOMPchSec}[1]{\rowcolor{colorAgua}\multicolumn{4}{|c|}{\bf #1}\\\hline}
\newcommand{\ESCOMPchItem}[2]{{\bf {#1}} & \multicolumn{3}{p{.66\textwidth}|}{#2}\\\hline}
\newcommand{\ESCOMPchSubItem}[3]{{\bf {#1}} & {#2} & \multicolumn{2}{p{.44\textwidth}|}{#3}\\\hline}
\newcommand{\ESCOMPchSubSubItem}[4]{{\bf {#1}} & {#2} & {#3}& {#4}\\\hline}

\cleardoublepage
{\centering{\Huge Contexto General Resumido}\bigskip\\}

\begin{longtable}{|p{.22\textwidth} |p{.22\textwidth} |p{.22\textwidth} |p{.22\textwidth} |}
\caption{Resumen del proyecto}
\label{tbl:projectCharter}\\
\hline

	\ESCOMPchItem{Proyecto:}{PetLife}
	\ESCOMPchItem{Responsable:}{PetLife Inc., Said Carbot}
	\ESCOMPchItem{Autoriza:}{PetLife Inc. Anibal Alejandro}
	\ESCOMPchItem{Background/Contexto:}{Existe una deficiencia crítica en la gestión administrativa y clínica de las PyMEs veterinarias en México, caracterizada por la fragmentación de datos y procesos manuales. El proyecto busca digitalizar estas clínicas para cumplir con la normativa vigente y aprovechar los datos históricos mediante IA.}
	\ESCOMPchItem{Beneficios esperados:}{
        \begin{itemize}
            \item Centralización del Expediente Clínico Electrónico (ECE).
            \item Reducción de pérdidas por merma mediante control de inventarios y alertas de caducidad.
            \item Apoyo a la medicina preventiva mediante predicción de esperanza de vida con IA.
        \end{itemize}
    }
	\ESCOMPchItem{Costo estimado:}{\$ 3,145,921.00 $\pm$ 10\%}
	\ESCOMPchSubSubItem{Fecha de inicio:}{Septiembre 2025}{\bf Fecha de término:}{Enero 2026}
	\ESCOMPchItem{Objetivo:}{Desarrollar un sistema web integral con componentes de inteligencia artificial para la administración clínica y operativa de consultorios veterinarios, optimizando el control del expediente, inventarios y seguimiento, proporcionando estimaciones sobre la esperanza de vida.}

	\ESCOMPchSec{Entregables Principales}
	\ESCOMPchSubItem{}{E1 -- Plataforma Web Integral}{Sistema de gestión clínica y administrativa desarrollado en Django (Backend) y HTML/JS (Frontend).}
	\ESCOMPchSubItem{}{E2 -- Base de Datos}{Repositorio relacional en PostgreSQL para la integridad de expedientes, usuarios e inventarios.}
	\ESCOMPchSubItem{}{E3 -- Módulo de IA}{Algoritmo predictivo entrenado para estimar la esperanza de vida basado en historial clínico y raza.}

	\ESCOMPchSec{Alcance del proyecto}
	\ESCOMPchItem{Incluye:}{
		\begin{Titemize}
			\Titem Gestión de roles: Administrativos, Veterinarios (Generales, Especialistas e Invitados) y Dueños.
			\Titem Expediente Clínico Electrónico (ECE) centralizado.
			\Titem Gestión de inventarios con alertas de stock mínimo y caducidad.
			\Titem Agenda inteligente con regla de cancelación automática (20 min de tolerancia).
			\Titem Visualización de citas en monitor de sala de espera.
			\Titem Módulo de predicción de esperanza de vida basado en IA.
            \Titem Notificaciones automáticas de medicina preventiva (vacunas/citas).
		\end{Titemize}
	}
	\ESCOMPchItem{Excluye:}{
		\begin{Titemize}
			\Titem Diagnósticos médicos definitivos (la IA es solo de apoyo/referencial).
			\Titem Integración con hardware de laboratorio (rayos X, análisis de sangre) en esta etapa.
			\Titem Telemedicina o consultas por videollamada.
		\end{Titemize}
	}

	\ESCOMPchItem{Criterio de éxito:}{El sistema debe gestionar el flujo completo de una clínica (cita-consulta-inventario) sin errores críticos y el modelo de IA debe ofrecer estimaciones coherentes con la literatura veterinaria.}
	\ESCOMPchItem{Metodología:}{SCRUM (Ágil) con sprints de desarrollo incremental y validación continua con usuarios finales.}

	\ESCOMPchSec{Datos de contacto}
	\ESCOMPchItem{Project Manager:}{Said Carbot, contacto institucional de PetLife.}
	\ESCOMPchItem{Project owner:}{Anibal Alejandro, contacto institucional de PetLife.}
	\ESCOMPchItem{Equipo de Desarrollo:}{López Hernández Julián, Mora Hernández Ángel Fernando.}

	\ESCOMPchItem{Riesgos y peligros:}{
		\begin{Titemize}
			\Titem Resistencia al cambio tecnológico por parte del personal veterinario.
			\Titem Inconsistencia en la captura de datos históricos para el entrenamiento de la IA.
            \Titem Fallas de conectividad en la clínica que impidan el acceso al sistema web.
		\end{Titemize}
	}
	\ESCOMPchItem{Supuestos:}{
		\begin{Titemize}
			\Titem La clínica cuenta con infraestructura de internet estable y equipos de cómputo.
			\Titem Los veterinarios adoptarán el uso del ECE en lugar del papel.
		\end{Titemize}
	}
	\ESCOMPchItem{Restricciones y dependencias:}{
		\begin{Titemize}
			\Titem Desarrollo obligatorio con Software Libre (Django/PostgreSQL).
			\Titem Cumplimiento de la NOM-012-ZOO-1993.
			\Titem Fecha límite de entrega para fase de pilotaje: 8 meses.
		\end{Titemize}
	}

	\ESCOMPchSec{Supervisión}
	\ESCOMPchSubItem{Juntas (Daily/Sprint):}{Equipo PetLife,}{Scrum Master}
	\ESCOMPchSubItem{Dudas Técnicas:}{Equipo de desarrollo,}{Líder Técnico}
	\ESCOMPchSubItem{Avances y Entregas:}{Stakeholders,}{Project Owner}

\end{longtable}