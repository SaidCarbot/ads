% !TeX root = proyecto.tex

%=========================================================
\chapter{Modelo de Persistencia}
\label{cap:modelo_persistencia}

Este capítulo final detalla la capa de datos del sistema **PetLife**. Se describe el esquema de almacenamiento relacional diseñado para garantizar la integridad de la información y se documentan las instrucciones SQL (Structured Query Language) que el sistema ejecuta durante los flujos descritos en el modelo dinámico.

%---------------------------------------------------------
\section{Descripción del modelo de persistencia}

El sistema implementa un modelo de persistencia relacional utilizando el Sistema Gestor de Base de Datos (SGBD) **MySQL**. La elección de esta tecnología responde a la necesidad de mantener relaciones fuertes entre las entidades (ej. Un Dueño $\to$ Muchas Mascotas) y garantizar las propiedades ACID (Atomicidad, Consistencia, Aislamiento y Durabilidad) en las transacciones.

La conexión desde el módulo web (PHP) se realiza mediante la interfaz **PDO (PHP Data Objects)**, lo que permite abstraer el acceso a datos y utilizar sentencias preparadas para mitigar riesgos de seguridad. Cabe destacar que, por eficiencia, los archivos binarios (imágenes de mascotas) no se almacenan en la base de datos; en su lugar, se persiste únicamente la ruta relativa del archivo (\texttt{VARCHAR}), mientras que el recurso físico reside en el sistema de archivos del servidor.

%---------------------------------------------------------
\section{Diseño de la base de datos}

El esquema de la base de datos se encuentra normalizado hasta la Tercera Forma Normal (3FN) para evitar redundancias. A continuación se presenta el Modelo Entidad-Relación (MER) que estructura las tablas del sistema.

\begin{figure}[h!]
    \centering
    %-------------------------------------------------------
    % INSTRUCCIONES PARA TU DIBUJO (Modelo Entidad-Relación):
    % Debes insertar aquí una imagen de tu diagrama ER o Modelo Relacional.
    % Si no tienes uno, dibuja al menos estas dos tablas conectadas:
    %
    % Tabla: usuarios
    % - id (PK, INT, AI)
    % - nombre (VARCHAR)
    % - email (VARCHAR, UNIQUE)
    % - password (VARCHAR) -> ¡Importante! Aquí se guarda el Hash
    % - rol (ENUM: 'admin', 'veterinario', 'usuario')
    %
    % Tabla: mascotas
    % - id (PK, INT, AI)
    % - nombre (VARCHAR)
    % - raza (VARCHAR)
    % - edad (INT)
    % - peso (FLOAT)
    % - foto_path (VARCHAR)
    % - id_usuario (FK) -> Conecta con usuarios.id
    %-------------------------------------------------------
    
    \framebox[0.9\textwidth]{\rule{0pt}{8cm} [Insertar aquí el Diagrama Entidad-Relación (DER)]}
    
    \caption{Diagrama de la base de datos `usuarios.sql` mostrando las tablas principales y sus relaciones.}
    \label{fig:der}
\end{figure}

%---------------------------------------------------------
\section{Comandos SQL utilizados}

A continuación, se listan las sentencias SQL fundamentales que corresponden a las operaciones identificadas en los Diagramas de Secuencia del Capítulo 6. Se utiliza la notación de parámetros (`?`) para indicar que se trata de sentencias preparadas seguras.

\subsection{Comandos SELECT (Lectura)}

Estas instrucciones se ejecutan cuando el usuario navega por el sistema o cuando el servicio de IA requiere datos para procesar una predicción.

\begin{itemize}
    \item \textbf{Inicio de Sesión:} Recuperación de credenciales para validar el acceso.
    \begin{verbatim}
    SELECT id, password, rol FROM usuarios WHERE email = ? LIMIT 1;
    \end{verbatim}

    \item \textbf{Listado de Mascotas:} Obtención de todas las mascotas pertenecientes al usuario en sesión (Visualización en \texttt{mis\_mascotas.php}).
    \begin{verbatim}
    SELECT id, nombre, raza, foto_path 
    FROM mascotas 
    WHERE id_usuario = ?;
    \end{verbatim}

    \item \textbf{Consulta para IA:} Obtención de datos clínicos específicos para enviar al modelo predictivo Python.
    \begin{verbatim}
    SELECT raza, edad, peso, historial_medico 
    FROM mascotas 
    WHERE id = ?;
    \end{verbatim}
\end{itemize}

\subsection{Comandos INSERT (Creación)}

Sentencias utilizadas en los procesos de registro de nuevos usuarios y alta de pacientes.

\begin{itemize}
    \item \textbf{Registro de Usuario:} Inserción de un nuevo actor en el sistema.
    \begin{verbatim}
    INSERT INTO usuarios (nombre, email, password, rol, fecha_registro) 
    VALUES (?, ?, ?, 'usuario', NOW());
    \end{verbatim}

    \item \textbf{Alta de Mascota:} Registro de un nuevo paciente vinculado a su dueño. Nótese el campo para la ruta de la imagen.
    \begin{verbatim}
    INSERT INTO mascotas (nombre, raza, edad, peso, foto_path, id_usuario) 
    VALUES (?, ?, ?, ?, ?, ?);
    \end{verbatim}
\end{itemize}

\subsection{Comandos UPDATE (Actualización)}

Operaciones ejecutadas cuando el usuario modifica información existente, como corregir datos de una mascota tras una consulta.

\begin{itemize}
    \item \textbf{Actualización de Expediente:} Modificación de datos variables como el peso o la edad.
    \begin{verbatim}
    UPDATE mascotas 
    SET edad = ?, peso = ? 
    WHERE id = ? AND id_usuario = ?;
    \end{verbatim}
\end{itemize}

\subsection{Comandos DELETE (Eliminado)}

Instrucciones para la eliminación de registros. Dependiendo de la configuración del sistema, esto puede ser un borrado físico o lógico.

\begin{itemize}
    \item \textbf{Baja de Mascota:} Eliminación del registro de la base de datos.
    \begin{verbatim}
    DELETE FROM mascotas 
    WHERE id = ? AND id_usuario = ?;
    \end{verbatim}
    \textit{Nota: El diseño contempla que, antes de ejecutar este comando SQL, el sistema de archivos debe eliminar la imagen física asociada en la carpeta \texttt{uploads/} para evitar archivos huérfanos.}
\end{itemize}