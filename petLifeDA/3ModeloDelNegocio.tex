% !TeX root = proyecto.tex

%=========================================================
\chapter{Modelo del Negocio}	
\label{cap:reqSist}

\cdtInstrucciones{Introduzca el capítulo describiendo el contenido del mismo, su organización y propósito.}

%---------------------------------------------------------
\section{Términos del Negocio}
\label{sec:terminosDeNegocio}

\cdtInstrucciones{En esta sección describa todos los términos del negocio que aparecen en la especificación del sistema.}
	
\begin{description}
	% Ejemplo de un término literal.
	\item[\hypertarget{tAutomovil}{Automóvil:}] ({\em es un tipo de \hyperlink{tVehiculo}{Vehículo}}) De cuatro ruedas con capacidad de 5 a 9 personas. 
	% Ejemplo de un término de entidad
	\item[\hypertarget{tCliente}{Cliente:}] Se refiere a todas las personas físicas y morales que \hyperlink{tRenta}{rentan} o han rentado un \hyperlink{tVehiculo}{vehículo}.
	
	\item[\hypertarget{tDirector}{Director:}] ({\em es un tipo de \hyperlink{tEmpleado}{Empleado}}) Es el empleado que tiene mayor rango de todos y no tiene superior, a diferencia de los demás.	
	\item[\hypertarget{tEmpleado}{Empleado:}] Se refiere a cualquier persona que labore en la empresa.
	
	\item[\hypertarget{tChecador}{Checador:}] ({\em Reloj asociado al atributo:} Hora de entrada y salida de un \hyperlink{tEmpleado}{empleado}. {\em Frecuencia de lectura:} Una vez al día para la entrada y otra para la salida durante los días laborales.
	
	\item[\hypertarget{tMotocicleta}{Motocicleta:}] ({\em es un tipo de {tVehiculo}{Vehículo}}) De dos ruedas con capacidad para una personas. 

	\item[\hypertarget{tRenta}{Renta:}] Se refiere al servicio que ofrece la empresa para prestar \hyperlink{tVehiculo}{vehículos} a los \hyperlink{tCliente}{clientes} por un tiempo definido.
	
	\item[\hypertarget{tVehiculo}{Vehiculo:}] Se refiere a los automóviles y motocicletas que la empresa usa para dar el servicio de renta a los \hyperlink{tCliente}{clientes}.
	
%	\brTermSensor{tVelocimetro}{Velocímetro:}{Velocidad de un Vehículo.}{Kilometros/hora.}{Constantemente siempre que el \cdtRef{tVehiculo}{vehículo} esté encendido.}
\end{description}

%----------------------------------------------------------
\section{Entidades del negocio (Modelo del dominio del problema)}
\label{sec:hechosDeNegocio}

\cdtInstrucciones{En esta sección describa todas las entidades del negocio y sus relaciones.}

	El modelo del dominio del problema se muestra en la figura~\ref{fig:modeloDeDominio}, a continuación se describen cada una de las entidades y sus relaciones.
	
\begin{figure}[htpb!]
	\begin{center}
		\includegraphics[angle=90,width=.95\textwidth]{images/modeloDelDominioDelProblema}
		\caption{Modelo del dominio del problema(O sea la Base de datos)}
		\label{fig:modeloDeDominio}
	\end{center}
\end{figure}

\begin{cdtEntidad}{Alumno}{Alumno}
	\brAttr{registro}{Registro}{Id}{Número de registro utilizado para identificar un alumno}{Sí}
	\brAttr{nombre}{Nombre}{Palabra Corta}
		{Nombre o nombres del alumno.}{Sí}
	\brAttr{primerApellido}{Primer apellido}{Palabra Corta}
		{Primer apellido del alumno.}{Sí}
	\brAttr{segundoApellido}{Segundo apellido}{Palabra Corta}
		{Segundo apellido del alumno.}{No}
	\brAttr{CURP}{CURP}{CURP}
		{CURP del alumno.}{Sí}
	\brAttr{nacimiento}{Nacimiento}{Fecha}
		{Fecha de nacimiento del alumno.}{Sí}
	\brAttr{genero}{Género}{Domicilio}
		{Género del alumno.}{No}
	\brAttr{telefono}{Teléfono}{Telefono}
		{Teléfono para contactar al alumno.}{Sí}
	\brAttr{correo}{Correo}{Correo}
		{Correo del alumno para enviar información académica y escolar y para recuperación de clave de acceso.}{Sí}
	\cdtEntityRelSection
	\brRel{\brRelComposition}{Domicilio}{Un \hyperlink{Alumno}{Alumno} reside en un \hyperlink{Domicilio}{Domicilio}}	
	\brRel{\brRelAgregation}{Grupo}{Un \hyperlink{Alumno}{Alumno} toma un \hyperlink{Curso}{Curso}}	
\end{cdtEntidad}

%- - - - - - - - - - - - - - - - - - - - - - - - - - - - - 
\begin{cdtEntidad}{AlumnoExtranjero}{Alumno Extranjero}%{}
	\brAttr{numeroResidente}{Numero de residente}{Id}{Número de registro dado por la Secretaría de Relaciones Exteriores a los extranjeros.}{Si}
	\brAttr{paisOrigen}{Pais origen}{\hyperlink{Pais}{País}}
		{País de origen del alumno extranjero.}{Sí}
	\cdtEntityRelSection
	\brRel{\brRelAgregation}{País}{Un \hyperlink{Alumno}{Alumno} es originario de un \hyperlink{Pais}{Pais}}	
	\brRel{\brRelGeneralization}{Alumno}{Un \hyperlink{AlumnoExtranjero}{Alumno Extranjero} es un  \hyperlink{Alumno}{Alumno}}	
	\brRel{\brRelParticipation}{Alumno}{Un \hyperlink{AlumnoExtranjero}{Alumno Extranjero} es un  \hyperlink{Alumno}{Alumno}}	
\end{cdtEntidad}

%---------------------------------------------------------
\section{Reglas de negocio}


\input{3-1-reglas}





