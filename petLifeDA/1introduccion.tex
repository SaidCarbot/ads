% !TeX root = proyecto.tex

%=========================================================
\chapter{Introducción}

Este documento presenta el análisis de requerimientos del proyecto ``{\em PetLife}'', el cual servirá como base para las actividades de análisis, diseño, construcción, pruebas y aceptación del sistema. El proyecto PetLife corresponde al desarrollo de una plataforma tecnológica orientada a la gestión integral de una clínica veterinaria, incorporando herramientas de Inteligencia Artificial como apoyo al seguimiento clínico de las mascotas.

El documento está dirigido a los integrantes del equipo de desarrollo, a los responsables del proyecto y a los posibles interesados en la implementación del sistema, con el objetivo de proporcionar una visión clara y estructurada del problema, de los usuarios involucrados y de las funcionalidades que deberá cumplir la solución propuesta.

%---------------------------------------------------------
\section{Presentación}

El proyecto PetLife surge ante la necesidad de mejorar la gestión de la información clínica y administrativa en clínicas veterinarias, así como de ofrecer a los dueños de mascotas un acceso más ágil y organizado a los servicios de atención médica. Actualmente, muchos de estos procesos se realizan de forma manual o mediante sistemas fragmentados, lo que dificulta el seguimiento del historial médico de las mascotas, la administración de citas y la toma de decisiones clínicas informadas.

En este contexto, PetLife propone el desarrollo de un sistema que permita el registro de dueños y mascotas, la gestión de citas y consultas médicas, así como el almacenamiento estructurado de información clínica relevante. Adicionalmente, el sistema contempla el uso de Inteligencia Artificial para apoyar el análisis de datos relacionados con la salud y la esperanza de vida de los animales, contribuyendo a diagnósticos más informados por parte del personal médico.

El propósito de este documento es describir de manera formal los requerimientos del sistema PetLife, estableciendo una referencia común para el desarrollo del proyecto. 

%---------------------------------------------------------
\section{Organización del contenido}

El documento se encuentra organizado en cinco capítulos principales, los cuales abordan de forma progresiva los distintos aspectos del proyecto.

En el Capítulo 1 se presenta una introducción general al proyecto, incluyendo la contextualización del problema, la justificación del sistema y la organización del documento.

El Capítulo 2, correspondiente al modelo del alcance, describe el análisis del problema, el contexto del negocio, la identificación de la problemática principal, sus causas y consecuencias, así como los objetivos generales y específicos del proyecto. En este capítulo también se detalla el plan de trabajo, la metodología utilizada, la definición de roles, los usuarios identificados y los requerimientos de usuario.

El Capítulo 3 aborda el modelo del negocio, donde se definen los términos relevantes del dominio, las entidades que conforman el sistema y las reglas de negocio que regulan su funcionamiento.

En el Capítulo 4 se presenta el modelo dinámico del sistema, incluyendo la identificación de los actores y la descripción detallada de los casos de uso, los cuales permiten comprender la interacción entre los usuarios y el sistema PetLife.

Finalmente, el Capítulo 5 corresponde al diseño de la interacción, donde se describe el mapa de navegación del sistema, el diseño de las interfaces de usuario y el diccionario de mensajes, proporcionando una visión clara de la experiencia de uso del sistema.
